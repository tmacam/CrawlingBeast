\documentclass[compress]{beamer}
%\mode<beamer>

% $Id$

\usepackage[utf8]{inputenc}
\usepackage[brazil]{babel}
\usepackage{graphicx}

\usetheme{Zurich}
\usepackage{tangocolors}
\usecolortheme{orchid}


% Informações sobre o documento
%%%%%%%%%%%%%%%%%%%%%%%%%%%%%%%%%%%%%%%%%%%%%%%%%%%%%%%%%%%%%%%%%%%%%%
\title[Modelos de Reputação]{Modelos de Reputação Baseados em Redes Sociais}
%\subtitle{UOL Bolsa Pesquisa 2006 - Workshop}
\author[Tiago Macambira]{Tiago Alves Macambira \\ Orientador: Dorgival
Olavo Guedes}
\institute{e-Speed \\ Departamento de Ciência da Computação \\ Universidade Federal de Minas Gerais}
\date{
UOL Bolsa Pesquisa 2006 - Workshop \\
4--5 de Abril de 2007}
\subject{Apresentação para o Workshop do UOL Bolsa Pesquisa 2006}
\logo{\includegraphics[scale=0.25]{speed}}


\begin{document}

\frame{ \titlepage }

\frame{ \frametitle{Roteiro} \tableofcontents}

\section{Introdução}
\subsection{Conceitos Iniciais}

    \begin{frame}{Conceitos Iniciais}
    \framesubtitle{Redes Sociais e Conceitos correlatos}
	\begin{block}{Redes Sociais}
	    Redes que modelam ou são produto da \alert{interação} humana.
	\end{block}
    \end{frame}

    \begin{frame}{Conceitos Iniciais}
    \framesubtitle{Redes Sociais e Conceitos correlatos}
	Como exemplo podemos citar diversos sistemas de comunicação
	\emph{on-line}:
	\begin{itemize}
	    \item Sítios tais como o Orkut, Friendster e UOL K
	    \item Listas de Discussão
	    \item Fóruns \emph{on-line}
	    \item Salas de bate-papo
	    \item Sistemas de \emph{Instant Messaging} (IRC, MSQ, Jabber)
	\end{itemize}
	
    \end{frame}

    \begin{frame}{Conceitos Iniciais}
    \framesubtitle{Mecanismos de auto-regulação}
	Nem todas as \alert{interações} entre seres humanos são benéficas para
	eles mesmos e para a comunidade.
        \pause

	Em comunidades reais, diversos mecanismos atuam
	para evitar que a própria comunidade se deteriore:
	\begin{itemize}[<alert@+|+->]
	    \item Identidade
	    \item Reputação
	    \item Regulamentação
	\end{itemize}
    \end{frame}


\subsection{Reputação e Moderação em Comunidades Virtuais}

    \begin{frame}{Reputação e Moderação em Comunidades Virtuais}
    \framesubtitle{Problemas no mapeamento do mundo real}
	Esses mecanismos não atuam em ambientes virtuais da
	mesma forma.
	\begin{itemize}
	    \item Identidade \pause
	        \begin{itemize}
                    \item Anonimato é muito comum em ambientes virtuais

		    \item Pessoas podem usar várias ``identidades
                    virtuais''
		\end{itemize}
	    \pause

	    \item Reputação \pause
	        \begin{itemize}
                    \item Nem todas as comunidades implementam
		    mecanismos de reputação

                    \item Anonimato pode tornar tais mecanismos inócuos
		\end{itemize}
	    \pause

	    \item Regulamentação (Moderação) \pause
	        \begin{itemize}
		    \item Regulamentação automatizada é muito limitada
		    \note{Riqueza semântica das interações humanas}

		    \item Regulamentação manual não escala com o tamanho
		    do número de usuários e corre o risco de se tornar
		    muito autoritária.
		\end{itemize}
	\end{itemize}
    \end{frame}

    \begin{frame}{Reputação e Moderação em Comunidades Virtuais}
    \framesubtitle{Mecanismos de Controle}

        Existem vários mecanismos para regulação em comunidades
        virtuais:
        \begin{itemize}
            \item \emph{Karma}
            \item \emph{White-listing}
            \item \emph{Grey-listing}
            \item Moderação simples (IRC, Salas de Bate-Papo)
            \item Convite
        \end{itemize}
        \pause

        Mas\ldots\pause
        \begin{itemize}[<|+->]
            \item se comunidades formam \alert{redes sociais}\ldots

            \item por que não usar \alert{informações da rede social} como parte
            do processo de \alert{regulamentação}\ldots

            \item de maneira \alert{automatizada}?
        \end{itemize}
    \end{frame}

\section{O Projeto}
\subsection{Objetivos}

    \begin{frame}{Objetivos}
	\begin{block}{Objetivos}
            Investigar mecanismos de \alert{moderação e reputação} que
            se aproveitem das características das \alert{redes sociais}
            que emergem de sistemas colaborativos \emph{on-line}.
	\end{block}
        \vspace{1em}
	\pause

        Proposta:
        \begin{itemize}
            \item Análise de \alert{interações} entre usuários e a comunidade e
            suas repercussões

	    \item Limitando-se à \alert{análise estrutural} das interações.

            \item Foco inicial limitado a fóruns \emph{on-line}
        \end{itemize}
    \end{frame}


\subsection{Aplicabilidade}

    \begin{frame}{Aplicabilidade}
	Isso pode ser usado para\ldots
        \begin{itemize}[<|+->]
            \item \alert{Atribuição automática de \emph{karma} em
            fóruns on-line}

            \item Indicação de \emph{flame-wars}, interações relevantes
            à moderação

	    \item \emph{Clipping} de notícias ou comentários em listas de discussão

            \item Detecção de ``robôs'' em bate-papos

            \item Indicação e promoção de editores em Wikis e outras
            mídias de edição colaborativa
	\end{itemize}
    \end{frame}

\subsection{Trabalhos Relacionados}

    \begin{frame}{Trabalhos Relacionados}
        Apesar de possuir um \alert{foco diferente}, nosso trabalho apresenta
	relação com as seguintes áreas e atividades:
	\begin{itemize}
	    \item Identificação de \emph{spam}
	    \item Classificação de texto
	\end{itemize}
    \end{frame}

\section{Metodologia}
\subsection{Visão de alto nível}

    \begin{frame}{Metodologia}
    \framesubtitle{Visão geral}
        Quais são os passos?
        \begin{itemize}
            \item Análise e seleção de sistemas de comunicação e
            comunidades \emph{on-line}
            \pause

            \item Monitoração e coleta de dados dos sistemas
            selecionados
            \begin{itemize}
                \item Coleta de dados
                \item \emph{Crawling}
                \item Análise passiva de tráfego
            \end{itemize}
            \pause

            \item Análise dos dados coletados
            \pause

            \item Simulação do modelo proposto
            \pause

            \item Implantação do modelo em um \emph{software}
        \end{itemize}
    \end{frame}

\subsection{Estudo de Caso: Slashdot}

    \begin{frame}{Estudo de Caso: Slashdot} %FIXME
        \begin{columns}
        \begin{column}{0.5\textwidth}
            \pgfdeclareimage[width=\textwidth]{cover}{snap_slashdot_cover}
            \pgfdeclareimage[width=\textwidth]{news}{snap_slashdot_news}
            \pgfdeclareimage[width=\textwidth]{comments}{snap_slashdot_comments}
            \pgfuseimage<2,5-|handout:1->{cover}
            \pgfuseimage<3|handout:0>{news}
            \pgfuseimage<4|handout:0>{comments}
        \end{column}

        \begin{column}{0.5\textwidth}
            O que é a Slashdot \pause
            \begin{itemize}[<alert@+|+->]
                \item Fórum
                \item Notícias
                \item Comentários
            \end{itemize}
            \pause

            Por que ela foi selecionada? \pause
            \begin{itemize}
                \item Popularidade

                \item Relevância

                \item Forma das interações

                \item Existência de mecanismos de Identificação, Reputação e
                Moderação
                \item Interações (comentários) são avaliadas e classificadas
            \end{itemize}
        \end{column}
        \end{columns}
    \end{frame}

\subsection{Coleta de Dados}

    \begin{frame}{Coleta de Dados}
        Complicações:
        \begin{itemize}
            \item Mecanismos de prevenção contra ataques de DDoS complicaram o
            processo
            \item 100+ nós do PlanetLab utilizados para realizar
            \emph{crawling} distribuído
        \end{itemize}

        Dados:
        \begin{itemize}
            \item 2 anos de notícias e comentários coletados
            \item 20 mil artigos
            \item 140 mil usuários
            \item 5 milhões de comentários
        \end{itemize}
    \end{frame}

\subsection{Resultados atuais}

    \begin{frame}{Resultados atuais}
        Slashdot:
        \begin{itemize}
            \item A rede de interações de seus usuários é uma rede
            com característica social
            \begin{itemize}
                \item Coeficiente de Agrupamento (CC): 0.13
                \item Diâmetro: 75
                \item Caminho Mínimo Médio ($l$): 3.13
            \end{itemize}
            \pause

            \item Sim, isso é esperado --- mas precisava ser comprovado.
        \end{itemize}

	Artigos:
        \begin{itemize}
            \item Dados utilizados em artigo para a ICWSM'07
            \item Artigo em elaboração (JBCS,?)
        \end{itemize}
    \end{frame}

\section{Conclusão}
\subsection{Estágio Atual}

    \begin{frame}{Conclusão}
    \framesubtitle{Estágio Atual}
        \begin{itemize}
            \item Algoritmos seqüenciais de análise social estão se
            mostrando ineficientes
            \pause
            \begin{itemize}
                \item Paralelizando vários dos algoritmos utilizados
            \end{itemize}
            \pause

            \item Estimativa de reputação dos usuários\pause

            \item Teste de hipóteses\pause
        \end{itemize}
    \end{frame}

\subsection{Passos Futuros}

    \begin{frame}{Conclusão}
    \framesubtitle{Passos futuros}
        \begin{itemize}
            \item Técnicas para avaliação das mensagens utilizando
            reputação dos usuários na vizinhança do contexto da
            mensagem.

            \item Averiguação da técnica em Wikis.
        \end{itemize}
    \end{frame}


\section{Perguntas}
\begin{frame}{Perguntas?}
\framesubtitle{Uma pausa para seu momento de epifania\ldots}

    Dúvidas?
    \vspace{1cm}
    \pause

    Sugestões?
    \vspace{1cm}
    \pause

    \begin{block}{Contato}
    \texttt{\{tmacam,dorgival\}@dcc.ufmg.br}
    \end{block}

\end{frame}

\end{document}

% vim:tw=72 syn=tex fileencoding=utf8 spelllang=pt spell autoindent:
% vim:softtabstop=4 smarttab expandtab shiftwidth=4:
